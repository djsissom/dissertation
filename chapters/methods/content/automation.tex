
%%%%%%%%%%%%%%%%%%%%%%%%%%%%%%%%%%%%%%%%%%%%%%%%%%%%%%%%%%%%%%%%%%%%%%%%%%%%%%%%
%
% Automation
%
%%%%%%%%%%%%%%%%%%%%%%%%%%%%%%%%%%%%%%%%%%%%%%%%%%%%%%%%%%%%%%%%%%%%%%%%%%%%%%%%

\section{Automation}
\label{sec:automation}

%%%%%%%%%%%%%%%%%%%%%%%%%%%%%%%%%%%%%%%%%%%%%%%%%%%%%%%%%%%%%%%%%%%%%%%%%%%%%%%%


Dealing with the large number of data files, programs, and pipeline steps used in our analysis quickly becomes prohibitive in terms of time and complexity when each must be dealt with completely ``by hand.''  In order to shorten the time needed for a full analysis of the data down to a reasonably human-scale level, a certain level of automation is required.  A combination of shell scripting and basic parallelization was used to this effect.  This has the added benefit of providing a self-documenting reproducibility to the analysis that was invaluable for the inevitable times when an error was discovered and the entire pipeline had to be re-run from the beginning.  In this section, we will give a very brief summary of the automation steps taken and the scripts written for these tasks.




