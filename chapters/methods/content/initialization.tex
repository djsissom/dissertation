
%%%%%%%%%%%%%%%%%%%%%%%%%%%%%%%%%%%%%%%%%%%%%%%%%%%%%%%%%%%%%%%%%%%%%%%%%%%%%%%%
%
% Simulation Initialization
%
%%%%%%%%%%%%%%%%%%%%%%%%%%%%%%%%%%%%%%%%%%%%%%%%%%%%%%%%%%%%%%%%%%%%%%%%%%%%%%%%

\section{Simulation Initialization}
\label{sec:initialization}

%%%%%%%%%%%%%%%%%%%%%%%%%%%%%%%%%%%%%%%%%%%%%%%%%%%%%%%%%%%%%%%%%%%%%%%%%%%%%%%%


We have already discussed the fundamentals of particle displacement with \za\ and \lpt\ in Section~\ref{subsubsec:computational_theory--perturbation_theory--particle_displacement}, so this section will instead provide an overview of the steps performed in the numerical implementation of simulation initialization.  The code used to generate \za\ and \lpt\ initial conditions for the simulations used in this study follows the prescription detailed in Appendix~D2 of \citet{1998MNRAS.299.1097S}, so we will simply summarize what is presented there.  For this section, a tilde will denote Fourier-space quantities.

Beginning with a linear power spectrum, a Gaussian density field $\tilde{\delta}(\mathbf{k})$, with wave number $\mathbf{k}$, is generated in Fourier space.  Equation~\ref{eq:particle_displacment--first_order_potential_poisson} is then used to find the Fourier space first-order potential $\tilde{\phi}^{(1)}(\mathbf{k})$, after which an inverse fast Fourier transform (FFT) is applied to produce $\phi^{(1)}(\mathbf{q})$.  The first-order particle displacements and velocities may then be found from Equations~\ref{eq:particle_displacement--za_displacement} and~\ref{eq:particle_displacement--za_velocity} by differencing $\phi^{(1)}(\mathbf{q})$ along the three coordinate vectors to obtain $\del_{\mathbf{q}} \phi^{(1)}$, providing the solution according to \za.

The \lpt\ solution may be found from blah.  The diagonal terms $\nabla_{11}^{2} \phi^{(1)}$, $\nabla_{22}^{2} \phi^{(1)}$, $\nabla_{33}^{2} \phi^{(1)}$ are obtained by diagonally differencing the components of the $\del_{\mathbf{q}} \phi^{(1)}$ array.  These are multiplied together to obtain the first term of Equation~\ref{eq:particle_displacement--second_order_overdensity}.  The non-diagonal terms $\phi_{,ij}^{(1)}(\mathbf{q}$ are found by differencing $\del_{\mathbf{q}} \phi^{(1)}$, and the results are accumulated to form the second term of Equation~\ref{eq:particle_displacement--second_order_overdensity}.  An FFT is applied to $\delta^{(2)}(\mathbf{q})$, Equation~\ref{eq:particle_displacment--second_order_potential_poisson} is solved in Fourier space, and an inverse FFT is applied to the resulting $\tilde{\phi^{(2)}}(\mathbf{k})$ to yield $\phi^{(2)}(\mathbf{q})$.  The second-order potential $\phi^{(2)}(\mathbf{q})$ is then differenced in each direction to yield $\del_{\mathbf{q}} \phi^{(2)}$.  With both $\del_{\mathbf{q}} \phi^{(1)}$ and $\del_{\mathbf{q}} \phi^{(2)}$, Equations~\ref{eq:particle_displacement--2lpt_displacement} and~\ref{eq:particle_displacement--2lpt_velocity} are used to find particle displacements and velocities, providing the solution for \lpt.



