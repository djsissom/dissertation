
%%%%%%%%%%%%%%%%%%%%%%%%%%%%%%%%%%%%%%%%%%%%%%%%%%%%%%%%%%%%%%%%%%%%%%%%%%%%%%%%
%
% Computational Theory
%
%%%%%%%%%%%%%%%%%%%%%%%%%%%%%%%%%%%%%%%%%%%%%%%%%%%%%%%%%%%%%%%%%%%%%%%%%%%%%%%%

\section{Computational Theory}
\label{sec:computational_theory}

%%%%%%%%%%%%%%%%%%%%%%%%%%%%%%%%%%%%%%%%%%%%%%%%%%%%%%%%%%%%%%%%%%%%%%%%%%%%%%%%


In this section, we present a broad overview of the fundamental theory and driving equations of computational astrophysics that are relevant to this work.  Specific code implementations, such as the \nbody\ simulation code \gadgettwo\ and the halo finder \rockstar, are discussed in Chapter~\ref{chap:methods}, so here we instead focus on the mathematical concepts that form the basis these codes rely on and have in common with varied other implementations.  Specifically, in this section, we discuss collisionless dynamics in \nbody\ simulations and simulation initialization with the Zel'dovich approximation (\za) and second-order Lagrangian perturbation theory (\lpt).  As the simulations used in our study are of collisionless dark matter only, we forgo a discussion of collisional hydrodynamics.




%~~~~~~~~~~~~~~~~~~~~~~~~~~~~~~~~~~~~~~~~~~~~~~~~~~~~~~~~~~~~~~~~~~~~~~~~~~~~~~~
\subsection{Collisionless Dynamics and \nbody\ Simulations}
\label{subsec:computational_theory--nbody_simulations}
%~~~~~~~~~~~~~~~~~~~~~~~~~~~~~~~~~~~~~~~~~~~~~~~~~~~~~~~~~~~~~~~~~~~~~~~~~~~~~~~


Astrophysical simulations of stars or dark matter, in essence, track a collisionless fluid, which is described in the continuum limit by the collisionless Boltzmann equation (CBE)
\begin{equation}
	\frac{\diff f(\mathbf{x}, \mathbf{v}, t)}{\diff t}
	\equiv \frac{\partial f}{\partial t} + \mathbf{v} \cdot \frac{\partial f}{\partial \mathbf{x}}
	- \frac{\partial \Phi}{\partial \mathbf{x}} \cdot \frac{\partial f}{\partial \mathbf{v}}
	= 0
\end{equation}
coupled to the Poisson equation
\begin{equation}
	\nabla^{2} \Phi(\mathbf{x}, t) = 4 \pi G \int f(\mathbf{x}, \mathbf{v}, t) \dd \mathbf{v}
\end{equation}
in an expanding background Universe, typically according to the Friedmann-Lema\^{i}tre-Robertson-Walker metric.  Here, $\Phi$ is the self consistent potential, and the distribution function $f(\mathbf{x}, \mathbf{v}, t)$ gives the mass density in phase space.  The high-dimensionality of the problem, however, makes directly solving the coupled system of equations intractable.  Instead, the \nbody\ method, in which the phase-space density is sampled with a finite number $N$ of tracer particles, is used to evolve the system in time.  For the following discussion, we primarily follow the notation in \citet{2005MNRAS.364.1105S}.

The system of particles is described by the Hamiltonian
\begin{equation}
	H(\mathbf{x}_{1}, \ldots, \mathbf{x}_{N}, \mathbf{p}_{1}, \ldots, \mathbf{p}_{N}, t)
	= \sum_{i} \frac{\mathbf{p}_{i}^{2}}{2 m_{i} a(t)^{2}} + \frac{1}{2} \sum_{ij} \frac{m_{i} m_{j} \varphi(\mathbf{x}_{i} - \mathbf{x}_{j})}{a(t)},
\end{equation}
where $\mathbf{x}_{i}$ are comoving coordinate vectors with corresponding canonical momenta $p_{i} = a^{2} m_{i} \mathbf{\dot{x}}_{i}$ and $a(t)$ is the time evolution of the scale factor that introduces explicit time dependence to the Hamiltonian.  For simulations with periodic boundary conditions, the interaction potential $\varphi(\mathbf{x})$ for a cube of size $L^{3}$ is the solution of
\begin{equation} \label{eq:computational_theory--nbody_simulations--discrete_poisson}
	\nabla^{2} \varphi(x) = 4 \pi G \left[ -\frac{1}{L^{3}} + \sum_{\mathbf{n}} \tilde{\delta}(\mathbf{x} - \mathbf{n}L) \right],
\end{equation}
where the sum over $\mathbf{n} = (n_{1}, n_{2}, n_{3})$ iterates over all integer triplets.  Here, the mean density is subtracted, and the dynamics of the system follow
\begin{equation}
	\nabla^{2} \phi(\mathbf{x}) = 4 \pi G [\rho(\mathbf{x}) - \bar{\rho}]
\end{equation}
with peculiar potential
\begin{equation}
	\phi(x) = \sum_{i} m_{i} \varphi(\mathbf{x} - \mathbf{x}_{i}).
\end{equation}
For non-periodic (vacuum) boundary conditions, the interaction potential for point masses simplifies to 
\begin{equation}
	\varphi(\mathbf{x}) = -\frac{G}{|\mathbf{x}|}
\end{equation}
for large separations.

At small particle separations as $|\mathbf{x}_{i} - \mathbf{x}_{j}| \rightarrow 0$, particle accelerations computed via the standard force law
\begin{equation}
	\mathbf{a}_{i} = -\sum_{j \ne i} \frac{G m_{j} | \mathbf{x}_{i} - \mathbf{x}_{j} |}{| \mathbf{x}_{i} - \mathbf{x}_{j} |^{3}}
\end{equation}
approach a numerical singularity that can introduce unphysical results for finite time-steps.  To avoid this scenario, numerical simulations employ a softening parameter $\epsilon > 0$ in the force law so that it does not diverge for small particle separations.  As a simple example, the softening parameter may be added to the particle displacement in the denominator of the Newtonian force law:
\begin{equation}
	\mathbf{F}_{i} = -\sum_{j \ne i} \frac{G m_{i} m_{j} | \mathbf{x}_{i} - \mathbf{x}_{j} |}{(| \mathbf{x}_{i} - \mathbf{x}_{j} |^{2} + \epsilon^{2})^{3/2}}.
\end{equation}
More generally, the single particle density distribution function $\tilde{\delta}(\mathrm{x})$ of Equation~\ref{eq:computational_theory--nbody_simulations--discrete_poisson} is the Dirac $\delta$-function convolved with a gravitational softening kernel of comoving scale $\epsilon$.  The specific choice of softening is dependent on the type of simulation and the system of study.  The softening parameter is typically on the order of the mean inter-particle separation.

Directly calculating forces for every particle from every other particle inherently requires a double sum, implying a computational cost of $\mathcal{O}(N^{2})$ algorithm complexity scaling.  For large $N$, this quickly becomes computationally expensive.  While the accuracy afforded by direct summation is sometimes necessary, such as for collisional systems like high-density star clusters, most studies can tolerate random force errors up to $\sim 1\%$ \citep{1993ApJ...402L..85H}, introducing the possibility of approximation methods.  There are a number of implementations for force approximations, but a typical result is a reduction of algorithmic complexity from $\mathcal{O}(N^{2})$ to $\mathcal{O}(N \log N)$.  The specific implementation employed by \gadgettwo\ is discussed in Section~\ref{subsec:gadget--gadget}.




%~~~~~~~~~~~~~~~~~~~~~~~~~~~~~~~~~~~~~~~~~~~~~~~~~~~~~~~~~~~~~~~~~~~~~~~~~~~~~~~
\subsection{Perturbation Theory and Particle Displacement}
\label{subsec:computational_theory--perturbation_theory}
%~~~~~~~~~~~~~~~~~~~~~~~~~~~~~~~~~~~~~~~~~~~~~~~~~~~~~~~~~~~~~~~~~~~~~~~~~~~~~~~


In order to retrieve reliable results from \nbody\ simulations, generation of accurate initial conditions for a given cosmology is imperative.  For cosmological simulations, the goal in creating initial conditions is to assign particle positions and velocities that are appropriate for a given simulation starting redshift $z_{\mathrm{start}}$ and consistent with the evolution of the Universe up to that point.

The subtle $\mathcal{O}(10^{-5})$ density perturbations in place at the CMB epoch are vulnerable to numerical noise and intractable to simulate directly.  Instead, a displacement field is applied to the particles to evolve them semi-analytically, nudging them from their initial positions to an approximation of where they should be at a more reasonable starting redshift for the numerical simulation.  Starting at a later redshift aids in avoiding interpolation systematics and round-off errors \citep{2007ApJ...671.1160L}.

For this discussion, we will assume a $\Lambda$CDM Universe, where the initial density distribution is described by a Gaussian random field defined by the power spectrum.  We wish to transform the information encoded in the power spectrum into a distribution of discrete particles at $z_{\mathrm{start}}$ that may then be evolved numerically.  The first step is to create a representation of the density field in Fourier space.  As the choice of power spectrum constrains the statistics of the density field and not its specific distribution, the specific realization of the field is generated from a random seed.  The typical procedure is to create a set of uniform random phases and assign amplitudes drawn from the Rayleigh distribution \citep{1985ApJS...57..241E}.  The density field may then be used as a basis for creating a particle distribution.

Beginning from a uniform lattice of Lagrangian positions, particles are displaced to new Eulerian positions and assigned velocities according to a displacement field $\boldsymbol{\Psi}$ that is derived from the density field.  The two most common methods for obtaining this displacement field are the Zel'dovich approximation \citep[\za,][]{1970A&A.....5...84Z} and second-order Lagrangian perturbation theory \citep[\lpt,][]{1994MNRAS.267..811B, 1994A&A...288..349B, 1995A&A...296..575B, 1998MNRAS.299.1097S}.  Initial conditions created with \za\ displace initial particle positions and velocities via a linear field \citep{1983MNRAS.204..891K, 1985ApJS...57..241E}, while \lpt\ initial conditions add a second-order correction term to the expansion of the displacement field \citep{1998MNRAS.299.1097S, 2005ApJ...634..728S, 2010MNRAS.403.1859J}.



%:::::::::::::::::::::::::::::::::::::::::::::::::::::::::::::::::::::::::::::::
\subsubsection{Particle Displacement with \za\ and \lpt}
\label{subsubsec:computational_theory--perturbation_theory--particle_displacement}
%:::::::::::::::::::::::::::::::::::::::::::::::::::::::::::::::::::::::::::::::


In this section, we give an overview of the equations necessary to generate initial conditions for \nbody\ simulations using \za\ and \lpt.  These results are fully described in Appendix~D1 of \citet{1998MNRAS.299.1097S}, and are largely reproduced here following that notation.

As mentioned above, our goal is to displace particles from their initial positions $\mathbf{q}$ to final Eulerian particle positions $\mathbf{x}$ via a displacement field $\boldsymbol{\Psi}(\mathbf{q})$:
\begin{equation}
	\mathbf{x} = \mathbf{q} + \boldsymbol{\Psi}(\mathbf{q}).
\end{equation}
If we define the conformal time $\tau = \int \diff t / a(t)$, where $a(t)$ is the scale factor, and the conformal expansion rate $\mathcal{H} \equiv \diff \ln a/\diff \tau = H a$, where $H$ is the Hubble constant, then the equation of motion for particle trajectories $\mathbf{x}(\tau)$ is given by
\begin{equation} \label{eq:particle_displacment--eq_of_motion}
	\frac{\diff^{2} \mathbf{x}}{\diff \tau^{2}} + \mathcal{H}(\tau) \frac{\diff \mathbf{x}}{\diff \tau} = -\del \Phi,
\end{equation}
where $\Phi$ is the gravitational potential and $\del$ is the gradient operator in Eulerian coordinates $\mathbf{x}$.  Using $1 + \delta(\mathbf{x}) = J^{-1}$, where $\delta(\mathbf{x}) \equiv [\rho(\mathbf{x},t) - \bar{\rho}] / \bar{\rho}$ is the density contrast and the Jacobian determinant is $J(\mathbf{q}, \tau) \equiv \det(\delta_{ij} + \boldsymbol{\Psi}_{i,j})$, where $\boldsymbol{\Psi}_{i,j} \equiv \partial \boldsymbol{\Psi}_{i} / \partial \mathbf{q}_{j}$, we may take the divergence of \ref{eq:particle_displacment--eq_of_motion} to obtain
\begin{equation} \label{eq:particle_displacement--div_eq_of_motion}
	J(\mathbf{q}, \tau) \del \cdot \left[ \frac{\diff^{2} \mathbf{x}}{\diff \tau^{2}} + \mathcal{H}(\tau) \frac{\diff \mathbf{x}}{\diff \tau} \right]
	= \frac{3}{2} \Omega \mathcal{H}^{2}(J - 1).
\end{equation}
Using $\del_{i} = (\delta_{ij} + \boldsymbol{\Psi}_{i,j})^{-1} \del_{\mathbf{q}_{j}}$, where the gradient operator in Lagrangian coordinates $\del_{\mathbf{q}} \equiv \partial / \partial \mathbf{q}$, this equation may be rewritten in terms of Lagrangian coordinates.

The solution to this transformed equation is given to first order by the Zel'dovich approximation:
\begin{equation} \label{eq:particle_displacement--zeldovich_approximation}
	\del_{\mathbf{q}} \cdot \boldsymbol{\Psi}^{(1)} = -D_{1}(\tau) \delta(\mathbf{q}),
\end{equation}
where $\delta(\mathbf{q})$ is the Gaussian density field determined by the initial conditions and $D_{1}(\tau)$ is the linear growth factor, which obeys
\begin{equation}
	\frac{\diff^{2} D_{1}}{\diff \tau^{2}} + \mathcal{H}(\tau) \frac{\diff D_{1}}{\diff \tau} = \frac{3}{2} \Omega \mathcal{H}^{2}(\tau) D_{1}.
\end{equation}
The Zel'dovich approximation solution for the particle displacement field is then given by
\begin{equation} \label{eq:particle_displacement--za_displacement}
	\mathbf{x}(\mathbf{q}, \tau) = \mathbf{q} + \boldsymbol{\Psi}(\mathbf{q}, \tau) \approx \mathbf{q} - D_{1}(\tau) \del \phi^{(1)}(\mathbf{q}),
\end{equation}
where $\phi^{(1)}(\mathbf{q})$ is a Lagrangian potential given by the initial conditions.  The velocities of particles initially at $\mathbf{q}$ are given by
\begin{equation} \label{eq:particle_displacement--za_velocity}
	\mathbf{v} \approx -D_{1}(\tau) \mathcal{H}(\tau) f \del \phi^{(1)}(\mathbf{q}),
\end{equation}
where $f(\Omega, \Lambda)$ is defined as
\begin{equation}
	f_{i}(\Omega, \Lambda) \equiv \frac{\diff \ln D_{i}}{\diff \ln a} = \frac{1}{\mathcal{H}} \frac{\diff \ln D_{i}}{\diff \tau}.
\end{equation}

The second-order (\lpt) correction is found by a perturbative solution to the non-linear equation for $\boldsymbol{\Psi}(\mathbf{q})$ (Equation~\ref{eq:particle_displacement--div_eq_of_motion} transformed to Lagrangian coordinates), expanding about the linear (\za) solution (Equation~\ref{eq:particle_displacement--zeldovich_approximation}) to yield \citep[e.g.,][]{1995A&A...296..575B}
\begin{equation}
	\del_{\mathbf{q}} \cdot \boldsymbol{\Psi}^{(2)}
	= \frac{1}{2} D_{2}(\tau) \sum_{i \ne j} \left[ \boldsymbol{\Psi}_{i,i}^{(1)} \boldsymbol{\Psi}_{j,j}^{(1)}
	- \boldsymbol{\Psi}_{i,j}^{(1)} \boldsymbol{\Psi}_{j,i}^{(1)} \right],
\end{equation}
where $D_{2}(\tau)$ is the second-order growth factor, which may be approximated as $D_{2}(\tau) \approx -3 D_{1}^{2}(\tau) / 7$ \citep{1995A&A...296..575B}.  The displacement field may then be written in terms of two Lagrangian potentials $\phi^{(1)}$ and $\phi^{(2)}$:
\begin{equation} \label{eq:particle_displacement--2lpt_displacement}
	\mathbf{x}(\mathbf{q}) = \mathbf{q} - D_{1} \del_{q} \phi^{(1)} + D_{2} \del_{q} \phi^{(2)}.
\end{equation}
Likewise, the comoving velocities are then given to second order by
\begin{equation} \label{eq:particle_displacement--2lpt_velocity}
	\mathbf{v} \equiv \frac{\diff \mathbf{x}}{\diff t} = - D_{1} f_{1} H \del_{q} \phi^{(1)} + D_{2} f_{2} H \del_{q} \phi^{(2)}.
\end{equation}
The logarithmic derivatives of the growth factors $f_{i}$ may be approximated as $f_{1} \approx \Omega^{5/9}$ and $f_{2} \approx 2 \Omega^{6/11}$ \citep{1995A&A...296..575B}.  The potentials $\phi^{(1)}$ and $\phi^{(2)}$ are derived by solving a pair of Poisson equations \citep{1994A&A...288..349B}:
\begin{equation} \label{eq:particle_displacment--first_order_potential_poisson}
	\nabla_{q}^{2} \phi^{(1)}(\mathbf{q}) = \delta^{(1)}(\mathbf{q}),
\end{equation}
\begin{equation} \label{eq:particle_displacment--second_order_potential_poisson}
	\nabla_{q}^{2} \phi^{(2)}(\mathbf{q}) = \delta^{(2)}(\mathbf{q}),
\end{equation}
where $\delta^{(1)}(\mathbf{q})$ is the linear overdensity, and $\delta^{(2)}(\mathbf{q})$ is the second-order overdensity given by
\begin{equation} \label{eq:particle_displacement--second_order_overdensity}
	\delta^{(2)}(\mathbf{q})
	= \sum_{i > j} \left\{ \phi_{,ii}^{(1)}(\mathbf{q}) \phi_{,jj}^{(1)}(\mathbf{q}) - \left[ \phi_{,ij}^{(1)}(\mathbf{q}) \right]^{2} \right\},
\end{equation}
where $\phi_{,ij}^{(n)} \equiv \partial^{2} \phi^{(n)} / \partial \mathbf{q}_{i} \partial \mathbf{q}_{j}$ \citep{2010MNRAS.403.1859J}.



%:::::::::::::::::::::::::::::::::::::::::::::::::::::::::::::::::::::::::::::::
\subsubsection{Transients and the Advantages of \lpt}
\label{subsubsec:computational_theory--perturbation_theory--transients}
%:::::::::::::::::::::::::::::::::::::::::::::::::::::::::::::::::::::::::::::::


A primary concern when generating cosmological initial conditions is the effects of non-linear decaying modes, or transients, which introduce deviations from the growing modes of the exact dynamics.  Linear growing modes of density and velocity perturbations are correctly reproduced by \za.  However, \za\ has shown to be inaccurate in regards to higher-order growing modes and non-linear correlations \citep{1987ApJ...320..448G, 1993ApJ...412L...9J, 1994ApJ...433....1B, 1994ApJ...426...14C, 1995ApJ...442...39J}, and fails to accurately represent statistical quantities that probe phase correlations of density and velocity fields \citep{1998MNRAS.299.1097S}.

We cannot expect accurate simulation results until enough time has passed for transients to have sufficiently decayed away.  Transients are damped proportional to $1 / a$ in \za.  In \lpt, however, transients are damped more quickly as $1 / a^{2}$.  Therefore, structure in \lpt\ should be accurate after fewer $e$-folding times than in \za\ \citep{1998MNRAS.299.1097S, 2006MNRAS.373..369C, 2010MNRAS.403.1859J}.  \citet{2013MNRAS.431.1866R} suggest that for \lpt-initialized simulations, between 10 and 50 expansion factors are needed before the relevant epoch of halo formation if percent level accuracy is to be achieved.

The practical result is that high-$\sigma$ DM density peaks at high redshift are suppressed in \za\ compared with \lpt\ for a given starting redshift \citep{2006MNRAS.373..369C}.  While differences in ensemble halo properties, such as the halo mass function, between simulation initialization methods are mostly washed away by $z=0$ \citep{1998MNRAS.299.1097S}, discrepancies between \za\ and \lpt\ remain at earlier redshifts \citep{2013MNRAS.431.1866R}, though these trends are relatively less studied \citep{2007ApJ...671.1160L}.



%:::::::::::::::::::::::::::::::::::::::::::::::::::::::::::::::::::::::::::::::
\subsubsection{Initial Redshift}
\label{subsubsec:computational_theory--perturbation_theory--initial_redshift}
%:::::::::::::::::::::::::::::::::::::::::::::::::::::::::::::::::::::::::::::::


When setting up an \nbody\ simulation, it is critical to choose an appropriate starting redshift, determined by box size and resolution \citep{2007ApJ...671.1160L}.  As \lpt\ more accurately displaces initial particle positions and velocities, initialization with \lpt\ allows for a later starting redshift compared with an equivalent \za-initialized simulation.  However, many \za\ simulations do not take this into account, starting from too late an initial redshift and not allowing enough $e$-foldings to adequately dampen away numerical transients \citep{2006MNRAS.373..369C, 2010MNRAS.403.1859J}.  In order to characterize an appropriate starting redshift, the relation between the initial rms particle displacement and mean particle separation must be considered.  The initial rms displacement $\Delta_{\mathrm{rms}}$ is given by
\begin{equation}
	\Delta_{\mathrm{rms}}^{2} = \frac{4 \pi}{3} \int_{k_{f}}^{k_{\mathrm{Ny}}} P(k, z_{\mathrm{start}}) \dd k,
\end{equation}
where $k_{f} = 2 \pi / L_{\mathrm{box}}$ is the fundamental mode, $L_{\mathrm{box}}$ is the simulation box size, $k_{\mathrm{Ny}} = \frac{1}{2} N k_{f}$ is the Nyquist frequency of an $N^{3}$ simulation, and $P(k, z_{\mathrm{start}})$ is the power spectrum at starting redshift $z_{\mathrm{start}}$.  In order to avoid the ``orbit crossings'' that reduce the accuracy of the initial conditions, $\Delta_{\mathrm{rms}}$ must be some factor smaller than the mean particle separation $\Delta_{p} = L_{\mathrm{box}} / N$ \citep{2012ApJ...761L...8H}.  For example, making orbit crossing a $\sim 10 \sigma$ event imposes $\Delta_{\mathrm{rms}} / \Delta_{p} = 0.1$.  However, for small-volume, high-resolution simulations, this quickly leads to impractical starting redshifts, placing such a simulation well into the regime of introducing errors from numerical noise caused by roundoff errors dominating the smooth potential.  A more relaxed requirement of $\Delta_{\mathrm{rms}} / \Delta_{p} = 0.25$, which makes orbit crossing a $\sim 4\sigma$ event, often proves a more practical choice.




