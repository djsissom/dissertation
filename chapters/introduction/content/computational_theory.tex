
%%%%%%%%%%%%%%%%%%%%%%%%%%%%%%%%%%%%%%%%%%%%%%%%%%%%%%%%%%%%%%%%%%%%%%%%%%%%%%%%
%
% Computational Theory
%
%%%%%%%%%%%%%%%%%%%%%%%%%%%%%%%%%%%%%%%%%%%%%%%%%%%%%%%%%%%%%%%%%%%%%%%%%%%%%%%%

\section{Computational Theory}
\label{sec:computational_theory}

%%%%%%%%%%%%%%%%%%%%%%%%%%%%%%%%%%%%%%%%%%%%%%%%%%%%%%%%%%%%%%%%%%%%%%%%%%%%%%%%


In this section, we present a broad overview of the fundamental theory and driving equations of computational astrophysics that are relevant to this work.  Specific code implementations, such as the \nbody\ simulation code \gadgettwo\ and the halo finder \rockstar, are discussed in Chapter~\ref{chap:methods}, so here we instead focus on the mathematical concepts that form the basis these codes rely on and have in common with varied other implementations.  Specifically, in this section, we discuss collisionless dynamics in \nbody\ simulations, simulation initialization with the Zel'dovich approximation (\za) and second-order Lagrangian perturbation theory (\lpt), and numerical definitions of dark matter halos.  As the simulations used in our study are of collisionless dark matter only, we forgo a discussion of collisional hydrodynamics.




%~~~~~~~~~~~~~~~~~~~~~~~~~~~~~~~~~~~~~~~~~~~~~~~~~~~~~~~~~~~~~~~~~~~~~~~~~~~~~~~
\subsection{Collisionless Dynamics and \nbody\ Simulations}
\label{subsec:computational_theory--nbody_simulations}
%~~~~~~~~~~~~~~~~~~~~~~~~~~~~~~~~~~~~~~~~~~~~~~~~~~~~~~~~~~~~~~~~~~~~~~~~~~~~~~~


Astrophysical simulations of stars or dark matter, in essence, track a collisionless fluid, which is described in the continuum limit by the collisionless Boltzmann equation (CBE)
\begin{equation}
	\frac{\diff f(\mathbf{x}, \mathbf{v}, t)}{\diff t}
	\equiv \frac{\partial f}{\partial t} + \mathbf{v} \cdot \frac{\partial f}{\partial \mathbf{x}}
	- \frac{\partial \Phi}{\partial \mathbf{x}} \cdot \frac{\partial f}{\partial \mathbf{v}}
	= 0
\end{equation}
coupled to the Poisson equation
\begin{equation}
	\nabla^{2} \Phi(\mathbf{x}, t) = 4 \pi G \int f(\mathbf{x}, \mathbf{v}, t) \dd \mathbf{v}
\end{equation}
in an expanding background Universe, typically according to the Friedmann-Lema\^{i}tre-Robertson-Walker metric.  Here, $\Phi$ is the self consistent potential, and the distribution function $f(\mathbf{x}, \mathbf{v}, t)$ gives the mass density in phase space.  The high-dimensionality of the problem, however, makes directly solving the coupled system of equations intractable.  Instead, the \nbody\ method, in which the phase-space density is sampled with a finite number $N$ of tracer particles, is used to evolve the system in time.  For the following discussion, we primarily follow the notation in \citet{2005MNRAS.364.1105S}.

The system of particles is described by the Hamiltonian
\begin{equation}
	H(\mathbf{x}_{1}, \ldots, \mathbf{x}_{N}, \mathbf{p}_{1}, \ldots, \mathbf{p}_{N}, t)
	= \sum_{i} \frac{\mathbf{p}_{i}^{2}}{2 m_{i} a(t)^{2}} + \frac{1}{2} \sum_{ij} \frac{m_{i} m_{j} \varphi(\mathbf{x}_{i} - \mathbf{x}_{j})}{a(t)},
\end{equation}
where $\mathbf{x}_{i}$ are comoving coordinate vectors with corresponding canonical momenta $p_{i} = a^{2} m_{i} \mathbf{\dot{x}}_{i}$ and $a(t)$ is the time evolution of the scale factor that introduces explicit time dependence to the Hamiltonian.  For simulations with periodic boundary conditions, the interaction potential $\varphi(\mathbf{x})$ for a cube of size $L^{3}$ is the solution of
\begin{equation} \label{eq:computational_theory--nbody_simulations--discrete_poisson}
	\nabla^{2} \varphi(x) = 4 \pi G \left[ -\frac{1}{L^{3}} + \sum_{\mathbf{n}} \tilde{\delta}(\mathbf{x} - \mathbf{n}L) \right],
\end{equation}
where the sum over $\mathbf{n} = (n_{1}, n_{2}, n_{3})$ iterates over all integer triplets.  Here, the mean density is subtracted, and the dynamics of the system follow
\begin{equation}
	\nabla^{2} \phi(\mathbf{x}) = 4 \pi G [\rho(\mathbf{x}) - \bar{\rho}]
\end{equation}
with peculiar potential
\begin{equation}
	\phi(x) = \sum_{i} m_{i} \varphi(\mathbf{x} - \mathbf{x}_{i}).
\end{equation}
For non-periodic (vacuum) boundary conditions, the interaction potential for point masses simplifies to 
\begin{equation}
	\varphi(\mathbf{x}) = -\frac{G}{|\mathbf{x}|}
\end{equation}
for large separations.

At small particle separations as $|\mathbf{x}_{i} - \mathbf{x}_{j}| \rightarrow 0$, particle accelerations computed via the standard force law
\begin{equation}
	\mathbf{a}_{i} = -\sum_{j \ne i} \frac{G m_{j} | \mathbf{x}_{i} - \mathbf{x}_{j} |}{| \mathbf{x}_{i} - \mathbf{x}_{j} |^{3}}
\end{equation}
approach a numerical singularity that can introduce unphysical results for finite time-steps.  To avoid this scenario, numerical simulations employ a softening parameter $\epsilon > 0$ in the force law so that it does not diverge for small particle separations.  As a simple example, the softening parameter may be added to the particle displacement in the denominator of the Newtonian force law:
\begin{equation}
	\mathbf{F}_{i} = -\sum_{j \ne i} \frac{G m_{i} m_{j} | \mathbf{x}_{i} - \mathbf{x}_{j} |}{(| \mathbf{x}_{i} - \mathbf{x}_{j} |^{2} + \epsilon^{2})^{3/2}}.
\end{equation}
More generally, the single particle density distribution function $\tilde{\delta}(\mathrm{x})$ of Equation~\ref{eq:computational_theory--nbody_simulations--discrete_poisson} is the Dirac $\delta$-function convolved with a gravitational softening kernel of comoving scale $\epsilon$.  The specific choice of softening is dependent on the type of simulation and the system of study.  The softening parameter is typically on the order of the mean inter-particle separation.

Directly calculating forces for every particle from every other particle inherently requires a double sum, implying a computational cost of $\mathcal{O}(N^{2})$ algorithm complexity scaling.  For large $N$, this quickly becomes computationally expensive.  While the accuracy afforded by direct summation is sometimes necessary, such as for collisional systems like high-density star clusters, most studies can tolerate random force errors up to $\sim 1\%$ \citep{1993ApJ...402L..85H}, introducing the possibility of approximation methods.  There are a number of implementations for force approximations, but a typical result is a reduction of algorithmic complexity from $\mathcal{O}(N^{2})$ to $\mathcal{O}(N \log N)$.  The specific implementation employed by \gadgettwo\ is discussed in Section~\ref{subsec:gadget--gadget}.




%~~~~~~~~~~~~~~~~~~~~~~~~~~~~~~~~~~~~~~~~~~~~~~~~~~~~~~~~~~~~~~~~~~~~~~~~~~~~~~~
\subsection{Perturbation Theory and Particle Displacement}
\label{subsec:computational_theory--perturbation_theory}
%~~~~~~~~~~~~~~~~~~~~~~~~~~~~~~~~~~~~~~~~~~~~~~~~~~~~~~~~~~~~~~~~~~~~~~~~~~~~~~~


Text goes here.



%:::::::::::::::::::::::::::::::::::::::::::::::::::::::::::::::::::::::::::::::
\subsubsection{The Zel'dovich Approximation}
\label{subsubsec:computational_theory--perturbation_theory--za}
%:::::::::::::::::::::::::::::::::::::::::::::::::::::::::::::::::::::::::::::::


Text goes here.



%:::::::::::::::::::::::::::::::::::::::::::::::::::::::::::::::::::::::::::::::
\subsubsection{Second-order Lagrangian Perturbation Theory}
\label{subsubsec:computational_theory--perturbation_theory--2lpt}
%:::::::::::::::::::::::::::::::::::::::::::::::::::::::::::::::::::::::::::::::


Text goes here.



%:::::::::::::::::::::::::::::::::::::::::::::::::::::::::::::::::::::::::::::::
\subsubsection{Particle Displacement with \za\ and \lpt}
\label{subsubsec:computational_theory--perturbation_theory--particle_displacement}
%:::::::::::::::::::::::::::::::::::::::::::::::::::::::::::::::::::::::::::::::


In this section, we give an overview of the equations necessary to generate initial conditions for \nbody\ simulations using \za\ and \lpt.  These results are fully described in Appendix~D1 of \citet{1998MNRAS.299.1097S}, and are largely reproduced here following that notation.

As mentioned above, our goal is to displace particles from their initial positions $\mathbf{q}$ to final Eulerian particle positions $\mathbf{x}$ via a displacement field $\boldsymbol{\Psi}(\mathbf{q})$:
\begin{equation}
	\mathbf{x} = \mathbf{q} + \boldsymbol{\Psi}(\mathbf{q}).
\end{equation}
If we define the conformal time $\tau = \int \diff t / a$, where $a(\tau)$ is the scale factor, and the conformal expansion rate $\mathcal{H} \equiv \diff \ln a/\diff \tau = H a$, where $H$ is the Hubble constant, then the equation of motion for particle trajectories $\mathbf{x}(\tau)$ is given by
\begin{equation} \label{eq:particle_displacment--eq_of_motion}
	\frac{\diff^{2} \mathbf{x}}{\diff \tau^{2}} + \mathcal{H}(\tau) \frac{\diff \mathbf{x}}{\diff \tau} = -\del \Phi,
\end{equation}
where $\Phi$ is the gravitational potential and $\del$ is the gradient operator in Eulerian coordinates $\mathbf{x}$.  Using the Poisson equation together with $1 + \delta(\mathbf{x}) = \mathbf{J}^{-1}$, where the Jacobian $\mathbf{J}(\mathbf{q}, \tau)$ is the determinant $\mathbf{J}(\mathbf{q}, \tau) \equiv \det(\delta_{ij} + \boldsymbol{\Psi}_{i,j})$, where $\boldsymbol{\Psi}_{i,j} \equiv \partial \boldsymbol{\Psi}_{i} / \partial \mathbf{q}_{j}$, we may take the divergence of \ref{eq:particle_displacment--eq_of_motion} to obtain
\begin{equation} \label{eq:particle_displacement--div_eq_of_motion}
	\mathbf{J}(\mathbf{q}, \tau) \del \cdot \left[ \frac{\diff^{2} \mathbf{x}}{\diff \tau^{2}} + \mathcal{H}(\tau) \frac{\diff \mathbf{x}}{\diff \tau} \right]
	= \frac{3}{2} \Omega \mathcal{H}^{2}(J - 1).
\end{equation}
Using $\del_{i} = (\delta_{ij} + \boldsymbol{\Psi}_{i,j})^{-1} \del_{\mathbf{q}_{j}}$, where the gradient operator in Lagrangian coordinates $\del_{\mathbf{q}} \equiv \partial / \partial \mathbf{q}$, this equation may be rewritten in terms of Lagrangian coordinates.

The solution to this transformed equation is given to first order by the Zel'dovich approximation:
\begin{equation} \label{eq:particle_displacement--zeldovich_approximation}
	\nabla_{\mathbf{q}} \cdot \boldsymbol{\Psi}^{(1)} = -D_{1}(\tau) \delta(\mathbf{q}),
\end{equation}
where $\delta(\mathbf{q})$ is the Gaussian density field determined by the initial conditions and $D_{1}(\tau)$ is the linear growth factor, which obeys
\begin{equation}
	\frac{\diff^{2} D_{1}}{\diff \tau^{2}} + \mathcal{H}(\tau) \frac{\diff D_{1}}{\diff \tau} = \frac{3}{2} \Omega \mathcal{H}^{2}(\tau) D_{1}.
\end{equation}
The Zel'dovich approximation solution for the particle displacement field is then given by
\begin{equation}
	\mathbf{x}(\mathbf{q}, \tau) = \mathbf{q} + \boldsymbol{\Psi}(\mathbf{q}, \tau) \approx \mathbf{q} - D_{1}(\tau) \del \phi^{(1)}(\mathbf{q}),
\end{equation}
where $\phi^{(1)}(\mathbf{q})$ is a Lagrangian potential given by the initial conditions.  The velocities of particles initially at $\mathbf{q}$ are given by
\begin{equation}
	\mathbf{v} \approx -D_{1}(\tau) \mathcal{H}(\tau) f \del \phi^{(1)}(\mathbf{q}),
\end{equation}
where $f(\Omega, \Lambda)$ is defined as
\begin{equation}
	f(\Omega, \Lambda) \equiv \frac{\diff \ln D}{\diff \ln a} = \frac{1}{\mathcal{H}} \frac{\diff \ln D}{\diff \tau}.
\end{equation}

The second-order (\lpt) correction is found by a perturbative solution to the non-linear equation for $\boldsymbol{\Psi}(\mathbf{q})$ (Equation~\ref{eq:particle_displacement--div_eq_of_motion} transformed to Lagrangian coordinates), expanding about the linear (\za) solution (Equation~\ref{eq:particle_displacement--zeldovich_approximation}) to yield \citep[e.g. ][]{1995A&A...296..575B}
\begin{equation}
	\nabla_{\mathbf{q}} \cdot \boldsymbol{\Psi}^{(2)}
	= \frac{1}{2} D_{2}(\tau) \sum_{i \ne j} \left[ \boldsymbol{\Psi}_{i,i}^{(1)} \boldsymbol{\Psi}_{j,j}^{(1)}
	- \boldsymbol{\Psi}_{i,j}^{(1)} \boldsymbol{\Psi}_{j,i}^{(1)} \right],
\end{equation}
where $D_{2}(\tau)$ is the second-order growth factor.

\begin{equation}
	D_{2}(\tau) \approx -\frac{3}{7} D_{1}^{2}(\tau) \Omega^{-2/63} \approx -\frac{3}{7} D_{1}^{2}(\tau)
\end{equation}

\begin{equation}
	D_{2}(\tau) \approx -\frac{3}{7} D_{1}^{2}(\tau) \Omega^{-1/143} \approx -\frac{3}{7} D_{1}^{2}(\tau)
\end{equation}

\begin{equation}
	\mathbf{x}(\mathbf{q}) = \mathbf{q} - D_{1} \nabla_{q} \phi^{(1)} + D_{2} \nabla_{q} \phi^{(2)}
\end{equation}

\begin{equation}
	\mathbf{v} \equiv \frac{\diff \mathbf{x}}{\diff t} = - D_{1} f_{1} H \nabla_{q} \phi^{(1)} + D_{2} f_{2} H \nabla_{q} \phi^{(2)}
\end{equation}

\begin{equation}
	f_{1} \approx \Omega^{3/5}, \;\;\;\; f_{2} \approx 2 \Omega^{4/7}
\end{equation}

\begin{equation}
	f_{1} \approx \Omega^{5/9}, \;\;\;\; f_{2} \approx 2 \Omega^{6/11}
\end{equation}

\begin{equation}
	\nabla_{q}^{2} \phi^{(1)}(\mathbf{q}) = \delta(\mathbf{q})
\end{equation}

\begin{equation}
	\nabla_{q}^{2} \phi^{(2)}(\mathbf{q})
	= \sum_{i > j} \left\{ \phi_{,ii}^{(1)}(\mathbf{q}) \phi_{,jj}^{(1)}(\mathbf{q}) - \left[ \phi_{,ij}^{(1)}(\mathbf{q}) \right]^{2} \right\}
\end{equation}




