
%%%%%%%%%%%%%%%%%%%%%%%%%%%%%%%%%%%%%%%%%%%%%%%%%%%%%%%%%%%%%%%%%%%%%%%%%%%%%%%%
%
% Introduction
%
%%%%%%%%%%%%%%%%%%%%%%%%%%%%%%%%%%%%%%%%%%%%%%%%%%%%%%%%%%%%%%%%%%%%%%%%%%%%%%%%
%
% Chapter Introduction
%
%%%%%%%%%%%%%%%%%%%%%%%%%%%%%%%%%%%%%%%%%%%%%%%%%%%%%%%%%%%%%%%%%%%%%%%%%%%%%%%%


Text goes here.  This is where we'll talk about the purpose of the project and the layout of this document.

The structure of this document is as follows:  The remainder of this chapter, Chapter~\ref{chap:introduction}, provides an introduction to the early universe and the processes that lead to galaxy-hosting dark matter halos, as well as the fundamentals of the computational theory for the numerical methods relevant to this discussion.  Chapter~\ref{chap:methods} examines in more detail the specific numerical methods used for this work, with emphasis on the methodologies of the codes themselves, how they are implemented in the context of the overall simulation and analysis pipeline, and the results obtained at each step.  Chapter~\ref{chap:2lpt} is a direct representation of the published paper which (more succinctly) presents an overview of the numerical methods and the main results in this work.  Chapter~\ref{chap:smbhs} is primarily the same material as previously submitted to fulfill the requirements of the Qualifying Exam, and is slightly edited to better suit the tone of this document.  Chapter~\ref{chap:conclusion} concludes with a discussion of the results in this work and the greater implications to the overall field.




