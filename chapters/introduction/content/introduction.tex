
%%%%%%%%%%%%%%%%%%%%%%%%%%%%%%%%%%%%%%%%%%%%%%%%%%%%%%%%%%%%%%%%%%%%%%%%%%%%%%%%
%
% Introduction
%
%%%%%%%%%%%%%%%%%%%%%%%%%%%%%%%%%%%%%%%%%%%%%%%%%%%%%%%%%%%%%%%%%%%%%%%%%%%%%%%%
%
% Chapter Introduction
%
%%%%%%%%%%%%%%%%%%%%%%%%%%%%%%%%%%%%%%%%%%%%%%%%%%%%%%%%%%%%%%%%%%%%%%%%%%%%%%%%


In this work, we explore the effects of simulation initialization technique on the properties of dark matter halo populations in the early Universe.  Specifically, we compare simulations initialized with the Zel'dovich approximation and second-order Lagrangian perturbation theory and measure the discrepancies in mass and concentration between halos in each simulation during the pre-reionization epoch.  Overall, we find that linear theory underestimates the growth of early halos, resulting in a suppressed halo mass distribution and large mass-dependent concentration fluctuations.  The first two chapters of this work are dedicated to introducing the underlying physics and numerical methods used in our research.  Our primary results are presented in the third chapter.

The structure of this document is as follows:  The remainder of this chapter, Chapter~\ref{chap:introduction}, provides an introduction to the early universe and the processes that lead to galaxy-hosting dark matter halos, as well as the fundamentals of the computational theory for the numerical methods relevant to this discussion.  Chapter~\ref{chap:methods} examines in more detail the specific numerical methods used for this work, with emphasis on the methodologies of the codes themselves, how they are implemented in the context of the overall simulation and analysis pipeline, and the results obtained at each step.  Chapter~\ref{chap:2lpt} is a direct representation of the paper submitted to the Astrophysical Journal (ApJ) on December 13, 2014, which (more succinctly) presents an overview of the numerical methods and the main results in this work.  Chapter~\ref{chap:smbhs} contains the material as previously submitted to fulfill the requirements of the Qualifying Exam and reviews supermassive black holes and their host galaxies.  Chapter~\ref{chap:conclusion} concludes with a review of the results in this work and the direction of future research.  Code for the various programs written for this work and used in our analysis is presented in the Appendices.




