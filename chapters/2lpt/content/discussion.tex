
%%%%%%%%%%%%%%%%%%%%%%%%%%%%%%%%%%%%%%%%%%%%%%%%%%%%%%%%%%%%%%%%%%%%%%%%%%%%%%%%
%
% Discussion
%
%%%%%%%%%%%%%%%%%%%%%%%%%%%%%%%%%%%%%%%%%%%%%%%%%%%%%%%%%%%%%%%%%%%%%%%%%%%%%%%%

\section{Discussion}
\label{sec:2lpt--discussion}

%%%%%%%%%%%%%%%%%%%%%%%%%%%%%%%%%%%%%%%%%%%%%%%%%%%%%%%%%%%%%%%%%%%%%%%%%%%%%%%%




%~~~~~~~~~~~~~~~~~~~~~~~~~~~~~~~~~~~~~~~~~~~~~~~~~~~~~~~~~~~~~~~~~~~~~~~~~~~~~~~
% Recapitulation
%~~~~~~~~~~~~~~~~~~~~~~~~~~~~~~~~~~~~~~~~~~~~~~~~~~~~~~~~~~~~~~~~~~~~~~~~~~~~~~~


As we evolve our DM halo population from our initial redshift to $z = 6$, we find that simulation initialization with \lpt\ can have a significant effect on halo population compared to initialization with \za.  The second order displacement boost of \lpt\ provides a head start on the initial collapse and formation of DM halos.  This head start manifests itself further along in a halo's evolution as more rapid growth and earlier mergers.  \lpt\ halos are, on average, more massive than their \za\ counterparts, with a maximum mean $\Delta M_{\mathrm{vir}}$ of $(9.3 \pm 1.2) \times 10^{-2}$ at $z = 15$.  The larger mass for \lpt\ halos is more pronounced for higher mass pairs, while \lpt\ halo concentration is larger on the small mass end.  Both mass and concentration differences trend towards symmetry about zero as halos evolve in time, with the smallest difference observed at the end of the simulations at $z = 6$, with a mean $\Delta M_{\mathrm{vir}}$ of $(1.79 \pm 0.31) \times 10^{-2}$.  Casual extrapolation of our observed trends with redshift to today would indicate that, barring structure like massive clusters that form at high redshift, \lpt\ and \za\ would produce very similar halo populations by $z = 0$.  However, the larger differences at high redshift should not be ignored.




%~~~~~~~~~~~~~~~~~~~~~~~~~~~~~~~~~~~~~~~~~~~~~~~~~~~~~~~~~~~~~~~~~~~~~~~~~~~~~~~
% Implications
%~~~~~~~~~~~~~~~~~~~~~~~~~~~~~~~~~~~~~~~~~~~~~~~~~~~~~~~~~~~~~~~~~~~~~~~~~~~~~~~


The earlier formation times and larger masses of halos seen in \lpt-initialized simulations could have significant implications with respect to early halo life during the Dark Ages.  Earlier forming, larger halos affect the formation of Pop-III stars, and cause SMBHs to grow more rapidly during their infancy \citep{2012ApJ...761L...8H}.  The epoch of peak star formation may also be shifted earlier.  This could additionally affect the contribution of SMBHs and early star populations to re-ionization.  Larger halos may also influence studies of the high-$z$ halo mass function, abundance matching, gas dynamics, AGN, clustering, and large scale structure formation.

In these discussions, it is important to note that it is wrong to assume that the \za\ halo properties are the ``correct'' halo properties, even in a statistical sense.  While halo mass suggests the most obvious shortcoming of \za\ simulations, even properties such as concentration---that show little difference on average between \lpt\ and \za---can have large discrepancies on an individual halo basis.  Failure to consider uncertainties in halo properties for high $z$ halos in \za\ simulations can lead to catastrophic errors.




%~~~~~~~~~~~~~~~~~~~~~~~~~~~~~~~~~~~~~~~~~~~~~~~~~~~~~~~~~~~~~~~~~~~~~~~~~~~~~~~
% Caveats
%~~~~~~~~~~~~~~~~~~~~~~~~~~~~~~~~~~~~~~~~~~~~~~~~~~~~~~~~~~~~~~~~~~~~~~~~~~~~~~~

We note a few caveats with our simulations and analysis.  We did not exclude substructure when determining the properties of a halo, and although this would not change the broad conclusions herein, care must be taken when comparing to works which remove subhalo particles in determining halo mass and concentration.  Halo matching is not perfect, as it is based on one snapshot at a time, and may miss count halos due to merger activity and differences in merger epochs.  However, we believe this effect to be minor.  While we compared \rockstar's output with our own fitting routines and found them to be in good agreement, \rockstar\ does not provide goodness of fit parameters for its NFW profile fitting and $R_{\mathrm{s}}$ measurements.  It also may be debated whether it makes sense to even consider concentration of halos at high redshift which are not necessarily fully virialized.

\rockstar\ does not provide goodness-of-fit parameters for its internal density profile measurements used to derive concentration, so error estimates for concentration values of individual halos are unknown.  Additionally, proper density profile fitting is non-trivial, as the non-linear interactions of numerical simulations rarely result in simple spherical halos that can be well described using spherical bins.

We use a simulation box size of only (10 Mpc)$^{3}$.  This is too small to effectively capture very large outlier density peaks.  We would, however, expect these large uncaptured peaks to most affected by \lpt\ initialization, so the effects presented here may even be dramatically underestimated.  Additionally, a larger particle number would allow us to consider smaller mass halos than we were able to here, and to better resolve all existing structure.  A higher starting redshift could probe the regime where \lpt\ initialization contributes the most.  It would also be of interest to evolve our halo population all the way to $z = 0$.  The addition of baryons in a fully hydrodynamical simulation could also affect halo properties.  These points may be address in future studies.




