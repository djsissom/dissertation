\section{Discussion}
\label{sec:discussion}

%\begin{enumerate}
%	\item Implications
%		\begin{enumerate}
%			\item Importance of using \lpt
%			\item Earlier mergers
%			\item More massive halos
%			\item Reionization/PopIII stars
%			\item SMBH formation head start
%			\item Epoch of peak star formation (shifted earlier?)
%			\item Extrapolation (by eye) of slope plot to later $z$
%			\item Abundance matching
%		\end{enumerate}
%	\item Limitations and caveats
%		\begin{enumerate}
%			\item Box size only 10 Mpc - not big enough for large outlier density peaks
%			\item Rockstar nondeterminism
%			\item Substructure
%			\item Halo matching not perfect - mergers at different times \textcolor{red}{Should probably also talk about this in analysis section.}
%			\item Statistics/goodness of fit on concentrations from rockstar
%			\item Fewer halos fit with python density profile code
%			\item Does concentration make sense at high $z$ (not virialized yet)?
%		\end{enumerate}
%	\item Future work
%		\begin{enumerate}
%			\item Bigger box
%			\item More particles
%			\item Start at earlier $z$?
%			\item Run to $z = 0$?
%			\item Other cosmologies?
%			\item Baryons?
%		\end{enumerate}
%\end{enumerate}

As we evolve our DM halo population from our initial redshift to $z = 6$, we find a number of trends and implications.  Simulation initialization with \lpt\ can have a large effect on halo population compared to initialization with \za.  The second order displacement boost of \lpt\ provides a head start on the initial collapse and formation of DM halos.  This head start manifests itself structurally further along in a halo's evolution as potentially earlier mergers, and a more relaxed core for a given timestep following a merger event.  \lpt\ halos are, on average, more massive and more concentrated than their \za\ counterparts.  The larger mass for \lpt\ halos is more pronounced for higher mass pairs, while \lpt\ halo concentration is larger on the small mass end.  Both mass and concentration differences trend towards symmetry about zero as halos evolve in time, with the smallest difference observed at the end of the simulations at $z = 6$.  Casual extrapolation of our observed trends with redshift to today would indicate that \lpt\ and \za\ would produce very similar halo populations by $z = 0$.  However, the larger differences at high redshift can have important consequences.

The use of \lpt\ for simulation initialization could have significant effects in various areas.  \lpt simulations can be started at a later initial redshift, saving computation time and reducing numerical noise effects of smooth low density regimes.  Many simulations are begun at too late a redshift for the size of box being simulated\cn, a problem which could be greatly reduced by initialization with \lpt.  Additionally, the earlier formation times and larger masses of \lpt\ halos could have important implications with respect to early halo life during the Dark Ages.  Earlier forming, larger halos could affect the formation of Pop-III stars, and cause super-massive black holes (SMBHs) to grow more rapidly during their infancy.  The epoch of peak star formation may also be shifted earlier.  This could additionally affect the contribution of SMBHs and early star populations to re-ionization.  Larger halos may also influence abundance matching studies.

We note a few caveats with our simulations and analysis.  We did not exclude substructure when determining the properties of a halo.  Halo matching is not perfect, as it is based on one snapshot at a time, and may miss count halos due to merger activity and differences in merger epochs.  However, we believe this effect to be minor.  While we compared \rockstar's output with our own fitting routines and found them to be in good agreement, \rockstar\ does not provide goodness of fit parameters for its NFW profile fitting and $R_{\mathrm{s}}$ measurements.  It may be debated whether it makes sense to even consider concentration of halos at high redshift which are not necessarily fully virialized.

We use a simulation box size of only 10 Mpc.  This is too small to effectively capture very large outlier density peaks.  We would, however, expect these large uncaptured peaks to be affected the most by \lpt\ initialization.  Additionally, a larger particle number would allow us to consider smaller mass halos than we were able to here, and to better resolve all existing structure.  A higher starting redshift could probe the regime where \lpt initialization contributes the most.  It would also be of interest to evolve our halo population all the way to $z = 0$.  The addition of baryons in a fully hydrodynamical simulation could also affect halo properties.  These may all be address in an upcoming set of simulations and follow--up paper.
