\section{Conclusion}
\label{sec:conclusion}


We analyzed three \lpt\ and \za\ simulation pairs and tracked the spherical overdensity dark matter halos therein with the 6-D phase space halo finder code \rockstar\ to compare the effect of initialization technique on properties of particle--matched dark matter halos.  This approach allowed us to directly compare matching halos between simulations and isolate the effect of using \lpt\ over \za.  In summary, we found the following:

\begin{itemize}
	\item \lpt\ halos get a head start in the formation process and grow faster than their \za\ counterparts.
	\item \lpt\ halos are, on average, more massive than \za\ halos.  Earlier collapse of the largest initial density peaks causes this effect to be most pronounced for the most massive halos.
	\item Halo concentration, on average, is similar for \lpt\ and \za\ halos.  However, halo pairs shift to more concentrated \lpt\ halos in the low mass regime and more concentrated \za\ halos in the high mass regime.
	\item Little difference, on average, is found between the average nuclear displacement of \lpt\ and \za\ halos.
	\item Differences in halo properties such as mass and concentration between \lpt\ and \za\ halos are most pronounced for high redshift.  The distributions begin to settle toward symmetry by the end of the simulations.
\end{itemize}

