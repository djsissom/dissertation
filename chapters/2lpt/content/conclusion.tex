
%%%%%%%%%%%%%%%%%%%%%%%%%%%%%%%%%%%%%%%%%%%%%%%%%%%%%%%%%%%%%%%%%%%%%%%%%%%%%%%%
%
% Conclusion
%
%%%%%%%%%%%%%%%%%%%%%%%%%%%%%%%%%%%%%%%%%%%%%%%%%%%%%%%%%%%%%%%%%%%%%%%%%%%%%%%%

\section{Conclusion}
\label{sec:conclusion}

%%%%%%%%%%%%%%%%%%%%%%%%%%%%%%%%%%%%%%%%%%%%%%%%%%%%%%%%%%%%%%%%%%%%%%%%%%%%%%%%


We analyzed three \lpt\ and \za\ simulation pairs and tracked the spherical overdensity dark matter halos therein with the 6-D phase space halo finder code \rockstar\ to compare the effect of initialization technique on properties of particle--matched dark matter halos from $z = 300$ to $z = 6$.  This approach allowed us to directly compare matching halos between simulations and isolate the effect of using \lpt\ over \za.  In summary, we found the following:

\begin{itemize}
	\item  \lpt\ halos get a head start in the formation process and grow faster than their \za\ counterparts.  Companion halos in \lpt\ and \za\ simulations may have offset merger epochs and differing nuclear morphologies.
	\item  \lpt\ halos are, on average, more massive than \za\ halos.  At $z = 15$, the mean of the $\Delta M_{\mathrm{vir}}$ distribution is $(9.3 \pm 1.2) \times 10^{-2}$, and 50\% of \lpt\ halos are at least 10\% more massive than their \za\ companions.  By $z = 6$, the mean $\Delta M_{\mathrm{vir}}$ is $(1.79 \pm 0.31) \times 10^{-2}$, and 10\% of \lpt\ halos are at least 10\% more massive.
	\item  This preference for more massive \lpt\ halos is dependent on redshift, with the effect most pronounced at high $z$.  This trend is best fit by $\Delta M_{\mathrm{vir}} = (7.88 \pm 0.17) \times 10^{-3} z - (3.07 \pm 0.14) \times 10^{-2}$.
	\item  Earlier collapse of the largest initial density peaks causes the tendency for more massive \lpt\ halos to be most pronounced for the most massive halos, a trend that increases with redshift. We find a trend of $\Delta M_{\mathrm{vir}} = (1.03 \pm 0.46) \times 10^{-2} \log(M_{\mathrm{vir,avg}}) - (2.6 \pm 3.8) \times 10^{-2}$ for $z = 10$.  By $z = 6$, this has flattened to $\Delta M_{\mathrm{vir}} = (3.49 \pm 0.99) \times 10^{-3} \log(M_{\mathrm{vir,avg}}) - (6.8 \pm 8.3) \times 10^{-3}$.  As a function of redshift, the slopes of these equations are fit by $\mathrm{Slope} = (9.4 \pm 2.4) \times 10^{-4} z - (1.8 \pm 1.8) \times 10^{-3}$.
	\item  Halo concentration, on average, is similar for \lpt\ and \za\ halos.  However, even by the end of the dark ages, the width of the $\Delta c$ distribution---$\sigma_{\Delta c} = 0.551 \pm 0.026$ at $z = 6$---is large and indicative of a significant percentage of halos with drastically mismatched concentrations, despite the symmetrical distribution of $\Delta c$.  At $z = 15$, 25\% of halo pairs have at least a factor of 2 concentration difference, with this falling to 15\% by $z = 6$.
	\item  There is a trend for \za\ halos to be more concentrated than \lpt\ halos at high mass.  We find $\Delta c = -(0.256 \pm 0.093) \log(M_{\mathrm{vir,avg}}) + (2.07 \pm 0.76)$ at $z = 15$ and $\Delta c = -(1.10 \pm 0.31) \times 10^{-2} \log(M_{\mathrm{vir,avg}}) - (0.103 \pm 0.026)$ at $z = 6$.  The slopes of these equations, as a function of redshift, are fit by $\mathrm{Slope} = -(7.3 \pm 1.9) \times 10^{-3} z + (3.7 \pm 1.4) \times 10^{-2}$.  This is not visible in the symmetrical $\Delta c$ distributions, as the trends are roughly centered about zero and are washed away when integrated across the entire mass range.
\end{itemize}

We have found that choice of initialization technique can play a significant role in the properties of halo populations during the pre-reionization dark ages.  The early halo growth displayed \lpt\ simulations, or conversely the delayed halo growth arising from the approximations made in \za-initialized simulations, makes careful attention to simulation initialization imperative, especially for studies of halos at high redshift.  It is recommended that future \nbody\ simulations be initialized with \lpt, and that previous high-$z$ or high-mass halo studies involving \za-initialized simulations be viewed with the potential offsets in halo mass and concentration in mind.

This work was conducted using the resources of the Advanced Computing Center for Research and Education (ACCRE) at Vanderbilt University, Nashville, TN.  We also acknowledge the support of the NSF CAREER award AST-0847696.  We would like to thank the referee for helpful comments, as well as the first author's graduate committee, who provided guidance throughout this work.




