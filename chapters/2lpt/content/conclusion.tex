\section{Conclusion}
\label{sec:conclusion}

We analyzed three \lpt\ and \za\ simulation pairs and tracked the spherical overdensity dark matter halos therein with the 6-D phase space halo finder code \rockstar\ to compare the effect of initialization technique on properties of particle--matched dark matter halos.  This approach allowed us to directly compare matching halos between simulations and isolate the effect of using \lpt\ over \za.  In summary, we found the following:

\begin{itemize}
	\item \lpt\ halos get a head start in the formation process and are often caught in a later stage of merging than their \za\ counterparts.  This leads to merging \za\ halos often appearing with a larger nuclear separation, more tidal features, and smaller masses than \lpt\ halos for a given snapshot.
	\item \lpt\ halos are, on average, more massive than \za\ halos.  Earlier collapse of the largest initial density peaks causes this effect to be most pronounced for the most massive halos.
	\item Halo concentration, likewise, is higher for \lpt\ halos than \za\ halos.  However, halo pairs shift to more concentrated \za\ halos in the high mass regime.
	\item Differences in nuclear displacement, mass, and concentration between \lpt\ and \za\ halos are most pronounced for high redshift.  Halo population mass and concentration difference distributions, especially, begin settling toward symmetry by the end of the simulations.
\end{itemize}

