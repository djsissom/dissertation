
%%%%%%%%%%%%%%%%%%%%%%%%%%%%%%%%%%%%%%%%%%%%%%%%%%%%%%%%%%%%%%%%%%%%%%%%%%%%%%%%
%
% Introduction
%
%%%%%%%%%%%%%%%%%%%%%%%%%%%%%%%%%%%%%%%%%%%%%%%%%%%%%%%%%%%%%%%%%%%%%%%%%%%%%%%%

\section{Introduction}
\label{sec:2lpt--introduction}

%%%%%%%%%%%%%%%%%%%%%%%%%%%%%%%%%%%%%%%%%%%%%%%%%%%%%%%%%%%%%%%%%%%%%%%%%%%%%%%%




%~~~~~~~~~~~~~~~~~~~~~~~~~~~~~~~~~~~~~~~~~~~~~~~~~~~~~~~~~~~~~~~~~~~~~~~~~~~~~~~
% Structure formation in the pre-reionization epoch
%~~~~~~~~~~~~~~~~~~~~~~~~~~~~~~~~~~~~~~~~~~~~~~~~~~~~~~~~~~~~~~~~~~~~~~~~~~~~~~~


The pre-reionization epoch is a time of significant evolution of early structure in the Universe.  Rare density peaks in the otherwise smooth dark matter (DM) sea lead to the collapse and formation of the first dark matter halos.  For example, at $z = 20$, $10^{7}~\mathrm{M}_{\odot}$ halos are $\sim 4\sigma$ peaks, and $10^{8}~\mathrm{M}_{\odot}$ halos, candidates for hosting the first supermassive black hole seeds, are $\sim 5\sigma$ peaks.

These early-forming dark matter halos provide an incubator for the baryonic processes that allow galaxies to form and transform the surrounding IGM.  Initial gas accretion can lead to the formation of the first Pop-III stars \citep{1986MNRAS.221...53C, 1997ApJ...474....1T, 2000ApJ...540...39A, 2002Sci...295...93A}, which, upon their death, can collapse into the seeds for supermassive black holes (SMBHs) \citep{2001ApJ...551L..27M, 2003MNRAS.340..647I, 2009ApJ...701L.133A, 2012ApJ...754...34J} or enrich the surrounding medium with metals through supernovae \citep{2002ApJ...567..532H, 2003ApJ...591..288H}.  The radiation from early quasars \citep{1987ApJ...321L.107S, 1999ApJ...514..648M, 2001AJ....122.2833F}, Pop-III stars \citep{1997ApJ...486..581G, 2003ApJ...584..621V, 2006ApJ...639..621A}, and proto-galactic stellar populations \citep{2012ApJ...752L...5B, 2012MNRAS.423..862K} all play a key role in contributing to re-ionizing the Universe by around $z = 6$ \citep{2001PhR...349..125B}.  Additionally, halo mergers can drastically increase the temperature of halo gas through shock heating, increasing X-ray luminosity \citep{2009MNRAS.397..190S} and unbinding gas to form the warm-hot intergalactic medium \citep{2008SSRv..134..141B, 2010MNRAS.405L..31S, 2012MNRAS.425.2974T}.




%~~~~~~~~~~~~~~~~~~~~~~~~~~~~~~~~~~~~~~~~~~~~~~~~~~~~~~~~~~~~~~~~~~~~~~~~~~~~~~~
% Current knowledge of high z mass, concentration, and density
%~~~~~~~~~~~~~~~~~~~~~~~~~~~~~~~~~~~~~~~~~~~~~~~~~~~~~~~~~~~~~~~~~~~~~~~~~~~~~~~


Since the pre-reionization era is such a critical epoch in galaxy evolution, much effort is expended to characterize the dark matter distribution accurately.  Statistical measures of the DM halo population, such as the halo mass function, are employed to take a census of the collapsed halos, while 3-point correlation functions are used to describe the clustering of these halos as a probe of cosmology.  Detailed analysis of the structure of individual halos involves characterizing the DM halo mass and density profile.

There are a number of ways to define a halo's mass, the subtleties of which become significant for mass-sensitive studies, such as the halo mass function \citep{1974ApJ...187..425P, 2007MNRAS.374....2R, 2006ApJ...642L..85H, 2007ApJ...671.1160L}.  For a review, see, e.g., \citet{2001A&A...367...27W} and references therein.  Additionally, see \citet{2005RvMP...77..207V} and references therein for a more observationally-focused discussion.  From a simulation standpoint, however, the two most common ways to obtain halo mass are through either spherical overdensity or friends-of-friends (FOF) techniques.  The spherical overdensity method identifies regions above a certain density threshold, either with respect to the critical density $\rho_{c} = 3 H^{2} / 8 \pi G$ or the background density $\rho_{b} = \Omega_{m} \rho_{c}$, where $\Omega_{m}$ is the matter density of the universe.  The mass is then the mass enclosed in a sphere of some radius with mean density $\Delta \rho_{c}$, where $\Delta$ commonly ranges from $\sim 100$ to $\sim 500$.  Alternatively, the FOF method finds particle neighbors and neighbors of neighbors defined to be within some separation distance \citep{1984MNRAS.206..529E, 1985ApJ...292..371D}.  Halo mass, then, is simply the sum of the masses of the linked particles.

The density profile of a DM halo is most often modeled with the NFW \citep{1996ApJ...462..563N} profile:
\begin{equation} \label{eq:nfw_profile}
	\rho(r) = \frac{ \rho_{0} }{ \frac{ r }{ R_{s}} \left( 1 + \frac{r}{R_{s}} \right)^{2} },
\end{equation}
where $\rho_{0}$ is the characteristic density, and the scale radius $R_{s}$ is the break radius between the inner $\sim r^{-1}$ and outer $\sim r^{-3}$ density profiles.  The NFW density profile is quantified by the halo concentration $c \equiv R_{\mathrm{vir}} / R_{s}$.  $R_{\mathrm{vir}}$ is the halo virial radius, which is often defined as the radius at which the average interior density is some factor $\Delta_{c}$ times the critical density of the universe $\rho_{c}$, where $\Delta_{c}$ is typically $\sim 200$.  Concentration may also be obtained for halos modeled with the Einasto \citep{1989A&A...223...89E} profile.  However, while halo profiles can be better approximated by the Einasto profile \citep{2004MNRAS.349.1039N, 2010MNRAS.402...21N, 2008MNRAS.387..536G}, the resulting concentrations display large fluctuations due to the smaller curvature of the density profile around the scale radius \citep{2012MNRAS.423.3018P}.

Generally, at low redshift, low mass halos are more dense than high mass halos \citep{1997ApJ...490..493N}, and concentration decreases with redshift and increases in dense environments \citep{2001MNRAS.321..559B}.  \citet{2007MNRAS.381.1450N} additionally find that concentration decreases with halo mass.  Various additional studies have explored concentration's dependence on characteristics of the power spectrum \citep{2001ApJ...554..114E}, cosmological model \citep{2008MNRAS.391.1940M}, redshift \citep{2008MNRAS.387..536G, 2011MNRAS.411..584M}, and halo merger and mass accretion histories \citep{2002ApJ...568...52W, 2003MNRAS.339...12Z, 2009ApJ...707..354Z}.  For halos at high redshift, \citet{2011ApJ...740..102K} find that concentration reverses and increases with mass for high mass halos, while \citet{2012MNRAS.423.3018P} additionally find that concentration's dependence on mass and redshift is better correlated with $\sigma(M,z)$, the rms fluctuation amplitude of the linear density field.




%~~~~~~~~~~~~~~~~~~~~~~~~~~~~~~~~~~~~~~~~~~~~~~~~~~~~~~~~~~~~~~~~~~~~~~~~~~~~~~~
% Simulation initialization
%~~~~~~~~~~~~~~~~~~~~~~~~~~~~~~~~~~~~~~~~~~~~~~~~~~~~~~~~~~~~~~~~~~~~~~~~~~~~~~~


Cosmological simulations that follow the initial collapse of dark matter density peaks into virialized halos often neglect to consider the nuances of initialization method.  Despite much effort in characterizing the resulting DM structure, comparatively less attention is paid to quantifying the effect of the initialization and simulation technique used to obtain the DM distribution.  The subtle $\mathcal{O}(10^{-5})$ density perturbations in place at the CMB epoch are vulnerable to numerical noise and intractable to simulate directly.  Instead, a displacement field is applied to the particles to evolve them semi-analytically, nudging them from their initial positions to an approximation of where they should be at a more reasonable starting redshift for the numerical simulation.  Starting at a later redshift saves computation time as well as avoiding interpolation systematics and round-off errors \citep{2007ApJ...671.1160L}.




%~~~~~~~~~~~~~~~~~~~~~~~~~~~~~~~~~~~~~~~~~~~~~~~~~~~~~~~~~~~~~~~~~~~~~~~~~~~~~~~
% 2LPT and ZA
%~~~~~~~~~~~~~~~~~~~~~~~~~~~~~~~~~~~~~~~~~~~~~~~~~~~~~~~~~~~~~~~~~~~~~~~~~~~~~~~


The two canonical frameworks for the initial particle displacement involved in generating simulation initial conditions are the Zel'dovich approximation \citep[\za,][]{1970A&A.....5...84Z} and 2nd-order Lagrangian Perturbation Theory \citep[\lpt,][]{1994MNRAS.267..811B, 1994A&A...288..349B, 1995A&A...296..575B, 1998MNRAS.299.1097S}.  \za\ initial conditions displace initial particle positions and velocities via a linear field \citep{1983MNRAS.204..891K, 1985ApJS...57..241E}, while \lpt\ initial conditions add a second-order correction term to the expansion of the displacement field \citep{1998MNRAS.299.1097S, 2005ApJ...634..728S, 2010MNRAS.403.1859J}.

Following \citet{2010MNRAS.403.1859J}, we briefly outline \lpt\ and compare it to \za.  In \lpt, a displacement field $\boldsymbol{\Psi}(\boldsymbol{q})$ is applied to the initial positions $\boldsymbol{q}$ to yield the Eulerian final comoving positions
\begin{equation} \label{eq:displacement}
	\boldsymbol{x} = \boldsymbol{q} + \boldsymbol{\Psi}.
\end{equation}
The displacement field is given in terms of two potentials $\phi^{(1)}$ and $\phi^{(2)}$:
\begin{equation} \label{eq:potentials}
	\boldsymbol{x} = \boldsymbol{q} - D_{1} \boldsymbol{\nabla}_{q} \phi^{(1)} + D_{2} \boldsymbol{\nabla}_{q} \phi^{(2)},
\end{equation}
with linear growth factor $D_{1}$ and second-order growth factor $D_{2} \approx -3 D_{1}^{2} / 7$.  The subscripts $q$ refer to partial derivatives with respect to the Lagrangian coordinates $\boldsymbol{q}$.  Likewise, the comoving velocities are given, to second order, by
\begin{equation} \label{eq:velocity}
	\boldsymbol{v} =  - D_{1} f_{1} H \boldsymbol{\nabla}_{q} \phi^{(1)} + D_{2} f_{2} H \boldsymbol{\nabla}_{q} \phi^{(2)},
\end{equation}
with Hubble constant $H$ and $f_{i} = \diff\, \mathrm{ln}\, D_{i} / \diff\, \mathrm{ln}\, a$, where $a$ is the expansion factor.  The relations $f_{1} \approx \Omega_{m}^{5/9}$ and $f_{2} \approx 2 \Omega_{m}^{6/11}$, with matter density $\Omega_{m}$, apply for flat models with a non-zero cosmological constant \citep{1995A&A...296..575B}.  The $f_{1}$, $f_{2}$, and $D_{2}$ approximations here are very accurate for most actual $\Lambda$CDM initial conditions, as $\Omega_{m}$ is close to unity at high starting redshift \citep{2010MNRAS.403.1859J}.  We may derive $\phi^{(1)}$ and $\phi^{(2)}$ by solving a pair of Poisson equations:
\begin{equation} \label{eq:poisson1}
	\nabla_{q}^{2} \phi^{(1)}(\boldsymbol{q}) = \delta^{(1)}(\boldsymbol{q}),
\end{equation}
with linear overdensity $\delta^{(1)}(\boldsymbol{q})$, and
\begin{equation} \label{eq:poisson2}
	\nabla_{q}^{2} \phi^{(2)}(\boldsymbol{q}) = \delta^{(2)}(\boldsymbol{q}).
\end{equation}
The second-order overdensity $\delta^{(2)}(\boldsymbol{q}$) is related to the linear overdensity field by
\begin{equation} \label{eq:second-order_overdensity}
	\delta^{(2)}(\boldsymbol{q}) = \sum_{i > j} \left\{ \phi_{,ii}^{(1)}(\boldsymbol{q}) \phi_{,jj}^{(1)}(\boldsymbol{q}) - \left[ \phi_{,ij}^{(1)}(\boldsymbol{q}) \right]^{2} \right\},
\end{equation}
where $\phi_{,ij} \equiv \partial^{2} \phi / \partial q_{i} \partial q_{j}$.  For initial conditions from \za, or first-order Lagrangian initial conditions, the $\phi^{(2)}$ terms of Equations~\ref{eq:potentials} and \ref{eq:velocity} are ignored.




%~~~~~~~~~~~~~~~~~~~~~~~~~~~~~~~~~~~~~~~~~~~~~~~~~~~~~~~~~~~~~~~~~~~~~~~~~~~~~~~
% Transients
%~~~~~~~~~~~~~~~~~~~~~~~~~~~~~~~~~~~~~~~~~~~~~~~~~~~~~~~~~~~~~~~~~~~~~~~~~~~~~~~


In theory, non-linear decaying modes, or transients, will be damped as $1 / a$ in \za.  In \lpt, however, transients are damped more quickly as $1 / a^{2}$.  It should be expected, then, that structure in \lpt\ will be accurate after fewer $e$-folding times than in \za\ \citep{1998MNRAS.299.1097S, 2006MNRAS.373..369C, 2010MNRAS.403.1859J}.  The practical result is that high-$\sigma$ DM density peaks at high redshift are suppressed in \za\ compared with \lpt\ for a given starting redshift \citep{2006MNRAS.373..369C}.  While differences in ensemble halo properties, such as the halo mass function, between simulation initialization methods are mostly washed away by $z=0$ \citep{1998MNRAS.299.1097S}, trends at earlier redshifts are less studied \citep{2007ApJ...671.1160L}.




%~~~~~~~~~~~~~~~~~~~~~~~~~~~~~~~~~~~~~~~~~~~~~~~~~~~~~~~~~~~~~~~~~~~~~~~~~~~~~~~
% In this paper...
%~~~~~~~~~~~~~~~~~~~~~~~~~~~~~~~~~~~~~~~~~~~~~~~~~~~~~~~~~~~~~~~~~~~~~~~~~~~~~~~


In this paper, we explore the effects of \za\ and \lpt\ on the evolution of halo populations at high redshift.  It is thought that \lpt\ allows initial DM overdensities to get a ``head start'' compared with \za, allowing earlier structure formation, more rapid evolution, and larger possible high-mass halos for a given redshift.  We explore this possibility by evolving a suite of simulations from $z = 300$ to $z = 6$ and comparing the resulting differences in halo properties arising from initialization with \za\ and \lpt\ in these these otherwise identical simulations.

We discuss the simulations, halo finding, and analysis methods in Section~\ref{sec:2lpt--methods}, results in Section~\ref{sec:2lpt--results}, implications, caveats, and future work in Section~\ref{sec:2lpt--discussion}, and a summary of our results and conclusions in Section~\ref{sec:2lpt--conclusion}.




