
%%%%%%%%%%%%%%%%%%%%%%%%%%%%%%%%%%%%%%%%%%%%%%%%%%%%%%%%%%%%%%%%%%%%%%%%%%%%%%%%
%
% Exploring Dark Matter Halo Populations in 2LPT and ZA Simulations
%
%%%%%%%%%%%%%%%%%%%%%%%%%%%%%%%%%%%%%%%%%%%%%%%%%%%%%%%%%%%%%%%%%%%%%%%%%%%%%%%%
%
% Chapter Introduction
%
%%%%%%%%%%%%%%%%%%%%%%%%%%%%%%%%%%%%%%%%%%%%%%%%%%%%%%%%%%%%%%%%%%%%%%%%%%%%%%%%


We study the structure and evolution of dark matter halos from $z = 300$ to $z = 6$ for two cosmological N-body simulation initialization techniques.  While the second order Lagrangian perturbation theory (\lpt) and the Zel'dovich approximation (\za) both produce accurate present day halo mass functions, earlier collapse of dense regions in \lpt\ can result in larger mass halos at high redshift.  We explore the differences in dark matter halo mass and concentration due to initialization method through three \lpt\ and three \za\ initialized cosmological simulations.  We find that \lpt\ induces more rapid halo growth, resulting in more massive halos compared to \za.  This effect is most pronounced for high mass halos and at high redshift, with a fit to the mean normalized difference between \lpt\ and \za\ halos as a function of redshift of $\mu_{\Delta M_{\mathrm{vir}}} = (7.88 \pm 0.17) \times 10^{3} z - (3.07 \pm 0.14) \times 10^{-2}$.  Halo concentration is, on average, largely similar between \lpt\ and \za, but retains differences when viewed as a function of halo mass.  For both mass and concentration, the difference between typical individual halos can be very large, even for symmetrically distributed quantities, highlighting the shortcomings of \za-initialized simulations for high-$z$ halo population studies.




