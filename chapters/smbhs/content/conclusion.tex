
\section{Conclusion}
\label{sec:conclusion}

We have seen that galaxies and the supermassive black holes at their centers both have their most dramatic periods of evolution around the same time.  Galaxy mergers grow both the galaxy and the SMBH.  Galaxies grow and become more elliptical as mergers bring in additional mass on orbits that can disrupt their gaseous disks.  These mergers also bring in counterpart supermassive black holes that fall toward the center of the galaxy and merge with the central SMBH, while also triggering accretion events and AGN feedback that pump energy back into the galaxy, shutting off star formation.



%%%%%%%%%%%%%%%%%%%%%%%%%%%%%%%%%%%%%%%%%%%%%%%%%%%%
\subsection{Correlations}

In light of these shared growth mechanisms, the correlations mentioned in Section \ref{sec:introduction} begin to move from a purely observational coincidence to a natural result of co-evolution.  The $M$--$\sigma$ relation is a natural byproduct of the simultaneous growth of supermassive black holes and their galaxies during merger events.  The mass of the SMBH increases due to the merging of binary companions and increased levels of accretion, while the host mass, and thus velocity dispersion, increases due to the infalling galaxy itself.  Likewise, the overabundance of AGN in galaxies lying in the green valley is the consequence of simultaneous change.  Mergers both trigger highly luminous AGN feedback and cause an inexorable shift from the blue cloud, through the green valley, to the red sequence.  Even the increase in scatter of the $M$--$\sigma$ relation at low masses can be explained by the galaxies having lower mass, and therefore being more likely to allow a gravitational wave recoil kicked black hole of a given velocity to escape.



%%%%%%%%%%%%%%%%%%%%%%%%%%%%%%%%%%%%%%%%%%%%%%%%%%%%
\subsection{Open Questions}

In the end, there remain a number of open questions.  How can very large supermassive black holes form so early?  What is dark matter actually made of?  How do galaxies retain their black holes if merger recoils can kick them with velocities greater than the escape velocity of the galaxy?  Over what range are our correlations truly valid?  These are just some of the questions that are currently being investigated, and promise to provide a rich field of study for years to come.



